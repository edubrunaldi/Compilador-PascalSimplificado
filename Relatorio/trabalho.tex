%%%%%%%%%%%%%%%%%%%%%%%%%%%%%%%%%%%%%%%%%%%%%%%%%%%%%%%%%%%
%		Trabalho 2 - Analisador Sintático
%
%	Linguagens de Programação e Compiladores	SCC0217
%	Prof Diego Raphael Amancio
%	
%	Eduardo Brunaldi dos Santos					8642515
%	Victor Luiz da Silva Mariano Pereira		8602444
%
%%%%%%%%%%%%%%%%%%%%%%%%%%%%%%%%%%%%%%%%%%%%%%%%%%%%%%%%%%%

% ---
% Definição do documento
% ---
\documentclass[
	% -- opções da classe memoir --
	12pt,				% tamanho da fonte
	openright,			% capítulos começam em pág ímpar (insere página vazia caso preciso)
	twoside,			% para impressão em recto e verso. Oposto a oneside
	a4paper,			% tamanho do papel.
]{abntex2}
% ---

% ---
% Pacotes básicos
% ---
\usepackage{fontspec}
\usepackage{polyglossia}
	\setdefaultlanguage{brazil}
\usepackage{lmodern}			% Usa a fonte Latin Modern
\usepackage{indentfirst}		% Indenta o primeiro parágrafo de cada seção.
\usepackage{color}				% Controle das cores
\usepackage{graphicx}			% Inclusão de gráficos
	\graphicspath{{figuras/}}
\usepackage{microtype}			% para melhorias de justificação
\usepackage{amssymb}
% ---

% ---
% Pacotes de citações
% ---
\usepackage[brazilian,hyperpageref]{backref}	% Paginas com as citações na bibl
\usepackage[alf]{abntex2cite}	% Citações padrão ABNT
\usepackage{caption}			% Pacote para resolver o problema do "listings", "minted" e códigos muito grandes
\usepackage{listings}			% Pacote para listagens de códigos
\usepackage{minted}			% Inserção de códigos fonte
% ---

% ---
% Pacotes para tabelas
% ---
\usepackage{longtable}
\usepackage{booktabs}
\usepackage{array}
% ---

% ---
% Pacotes adicionais
% ---
%\usepackage{showframe}			% Pacote para mostrar as caixas de texto
% ---

% ---
% Informações de dados para CAPA e FOLHA DE ROSTO
% ---
\titulo{Analisador Sintático}
\autor{Eduardo Brunaldi dos Santos				--- 8642515\and
		\\ Victor Luiz da Silva Mariano Pereira	--- 8602444}
\local{Brasil}
\data{2017}
\instituicao{%
  Universidade de São Paulo -- USP
  \par
  Instituto de Ciências Matemáticas e de Computação -- ICMC
  \par
  Linguagens de Programação e Compiladores -- SCC0217}
\tipotrabalho{Trabalho Acadêmico}
% ---

% ---
% Configurações de aparência do PDF final

% alterando o aspecto da cor azul
\definecolor{blue}{RGB}{41,5,195}

% informações do PDF
\makeatletter
\hypersetup{
	%pagebackref=true,
	pdftitle={\@title},
	pdfauthor={\@author},
	pdfsubject={Linguagens de Programação e de Compiladores},
	pdfcreator={LuaLaTeX with abnTeX2},
	pdfkeywords={USP }{ICMC }{Compiladores }{Análise sintática }{LALG },
	colorlinks=true,			% false: boxed links; true: colored links
	linkcolor=blue,				% color of internal links
	citecolor=blue,				% color of links to bibliography
	filecolor=magenta,			% color of file links
	urlcolor=blue,
	bookmarksdepth=4
}
\makeatother
% ---

% ---
% Espaçamentos entre linhas e parágrafos
% ---

% O tamanho do parágrafo é dado por:
\setlength{\parindent}{1.3cm}

% Controle do espaçamento entre um parágrafo e outro:
\setlength{\parskip}{0.2cm}  % tente também \onelineskip

% ---
% compila o indice
% ---
\makeindex
% ---

% ----
% Início do documento
% ----
\begin{document}

% Retira espaço extra obsoleto entre as frases.
\frenchspacing

% ----------------------------------------------------------
% ELEMENTOS PRÉ-TEXTUAIS
% ----------------------------------------------------------
\pretextual

% ---
% Capa
% ---
\imprimircapa
% ---

% ---
% Folha de rosto
% (o * indica que haverá a ficha bibliográfica)
% ---
\imprimirfolhaderosto*
% ---

% ---
% RESUMOS
% ---
% resumo em português
%\setlength{\absparsep}{18pt} % ajusta o espaçamento dos parágrafos do resumo
%\begin{resumo}
%\end{resumo}

% ---
% inserir lista de ilustrações
% ---
\pdfbookmark[0]{\listfigurename}{lof}
\listoffigures*
\clearpage
% ---

% ---
% inserir lista de tabelas
% ---
%\pdfbookmark[0]{\listtablename}{lot}
%\listoftables*
%\clearpage
% ---

% ---
% inserir lista de códigos fonte
% ---
\renewcommand\listoflistingscaption{Lista de Códigos Fonte}
\listoflistings
\clearpage
%---

% ---
% inserir o sumario
% ---
\pdfbookmark[0]{\contentsname}{toc}
\tableofcontents*
\clearpage
% ---

% ----------------------------------------------------------
% ELEMENTOS TEXTUAIS
% ----------------------------------------------------------
\textual

% ----------------------------------------------------------
% Corpo do trabalho
% ----------------------------------------------------------

%\include{introducao}
%\include{decisoes}
%\include{instrucoes}
%\include{testes}

% ----------------------------------------------------------
% ELEMENTOS PÓS-TEXTUAIS
% ----------------------------------------------------------
\postextual
% ----------------------------------------------------------

% ----------------------------------------------------------
% Referências bibliográficas
% ----------------------------------------------------------
%\bibliography{trabalho}

\end{document}
